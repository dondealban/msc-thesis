%--------------------------------------------------------------------------
% Conclusions and Recommendations
%--------------------------------------------------------------------------

\chapter{Conclusions and Recommendations}
\label{cha: conclusions}

This study found multi-year ALOS/PALSAR mosaic data to be temporally consistent and capable of classifying between forest and non-forest areas; hence making it suitable tool for mapping and monitoring forest extent. Unlike fully polarimetric SAR data which require more complex image processing, the mosaic data is less complicated and is provided to the public free of charge, which makes it a viable data source for operational, wall-to-wall forest mapping and monitoring. In terms of discriminating forest types, ALOS/PALSAR was not suitable for discriminating between forest types based on canopy characteristics or a forest typology such as the FAO Global Forest Resource Assessment classification scheme.

Cluster analysis provided information on structure and hierarchy of forest types in aid of classification. The analysis can similarly be applied for other classification systems to reveal hierarchy and structure of land cover types, which can guide subsequent classification processes. Decision trees were successfully used in past research for classifying land and forest cover types using SAR data, and even outperformed other statistical classification techniques. A decision tree classifier was also applied in this study, which provided information on classification accuracies to assess the degree to which different forest types can be discriminated using a combination of feature attributes. A combination of hierarchical cluster analysis and decision tree classification techniques is recommended as a viable approach for determining forest hierarchy by finding structure in the data and using the output dendrogram to guide the subsequent decision tree classification.

However, while the multi-level hierarchical classification approach is recommended, low classification accuracies (or high error rates) were obtained, which may be primarily due to the incompatibility of the sensitivities of the PALSAR sensor to forest structural attributes with the forest classification scheme used, particularly the forest types of the FAO Global Forest Resources Assessment. Hence, a different forest classification system such as the one devised by Fernando et al. (2008), which described Philippine forest formations typified based on their physical attributes, particularly physiognomy, elevation, and soil substrate may be a explored as a plausible forest type classification scheme, and a topic of subsequent research for mapping Philippine forest cover types using SAR data.

An object-based approach also allowed the use of multiple sources of data to increase the dimensionality of the input data for classification. Topographic and texture feature attributes helped to decrease classification error rates in contrast to using solely polarimetric attributes for classifying land and forest cover types. In turn, the availability of multiple data sources was maximised by the decision tree classifier through its selection of the most parsimonious feature attributes to improve the classification outcome. A subject of future research can be to test the sensitivity of specific texture measures to forest type classification, and exploring the effects of varying image segmentation parameters in terms of assessing temporal consistency, multi-level classification hierarchy, and classification accuracy.

Discriminating and mapping forest cover types is a key piece of information to support forest management. Combining multi-frequency SAR data such as C-band and L- band radar sensors is suggested as the subject of future research to improve the discrimination of forest types. C-band data such as Sentinel-1A sensor, which is sensitive to forest canopy components can complement L-band PALSAR data, which is more sensitive to forest structure components. Since SAR overcomes limitations of optical data, combining the data from these two SAR platforms have the potential to provide spatially explicit information in support of an operational, wall-to-wall forest mapping and monitoring system.
