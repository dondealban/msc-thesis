%--------------------------------------------------------------------------
% Discussion
%--------------------------------------------------------------------------

\chapter{Discussion}
\label{cha: discussion}

Sustainable forest management needs accurate information on the extent of forests, of which remote sensing is an effective tool for mapping and monitoring of forest cover. Land and forest cover maps of the Philippines in the past were mainly generated using optical data, but spaceborne radar sensors have been promising as a viable and accurate data source. This study evaluated the suitability of ALOS/PALSAR mosaic data for mapping and monitoring of forest cover types in the Philippines by assessing the following indicators, particularly: the temporal consistency of multi-year PALSAR mosaic data; the presence and consistency of a multi-level classification hierarchy based on a combination of polarimetric, topographic, and texture variables; and accuracy of forest type classification using decision trees.

This study demonstrated the temporal consistency of globally available PALSAR mosaic data. This makes ALOS/PALSAR data suitable for regular and periodic forest monitoring purposes, appropriate for developing operational national forest monitoring systems, and engaging in performance-based incentive schemes such as Reducing Emissions from Deforestation and Forest Degradation (REDD+), wherein data consistency is a key requirement. This consistency was demonstrated by analysing the distribution of radar backscatter vis-a-vis data acquisition dates of PALSAR mosaic data acquired in a single year but the data strips used to put the mosaic together were taken at different seasons, and of PALSAR mosaic data acquired along the same data strip or path but at different years.

The existence of a multi-level classification hierarchy of forest types developed using hierarchical cluster analysis was also demonstrated in the study, which was consistent across annual PALSAR mosaics. This was demonstrated by high Baker’s Gamma indices, which indicated that the hierarchical clustering of forest types were similar and consistent across years, and reinforced the temporal consistency of PALSAR mosaic data. Using polarimetric variables showed the highest consistencies between pairwise dendrograms. The inclusion of topographic or texture variables\textemdash{}while still resulting into high indices\textemdash{}suggested that the variables marginally decreases the consistency of the multi-level classification hierarchy over multi-year PALSAR mosaics. The most consistent multi-level classification hierarchy of forest types guided subsequent forest type classification using decision trees.

Hierarchical multi-level classification provided a suitable approach for quantifying the degree of accuracy in discriminating different forest cover types. Employing a decision tree classification approach identified the most relevant image layers and feature attributes for classifying different classes. Combining polarimetric, topographic, and texture attributes yielded lowest classification error rates compared to use of polarimetric data alone, or the combination of either polarimetric and topographic or polarimetric and texture variables in the classification.

However, even with the best combination of feature attributes used in study, high error rates were obtained in the classification of forest types following the multi-level hierarchical classification. In turn, forest cover types were mapped incorrectly in the resulting annual forest cover maps, and area statistics varied considerably among forest types across years.

In the separate assessment following a heuristic classification approach, misclassification error rates were lower in separating closed and open canopy forests compared to separating between broadleaved, coniferous, and mixed forests. This was in contrast to the findings of Dobson et al. \citeyearpar{dobson_knowledge-based_1996}, particularly in their investigation of the capability of ERS-1 and JERS-1 SAR imagery using a knowledge-based classification with hierarchical decision rules to differentiate structural land cover categories, of which they concluded that L-band was capable of differentiating forest types based on their structural characteristics, particularly as either broadleaved or coniferous, but not canopy attributes like closed or open canopy cover.

Also, within the heuristic approach, the inclusion of topographic variables in either forest groups affected the classification accuracy by yielding lower error rates as compared to including texture variables, which had no effect on improving classification accuracy and was no different than using polarimetric variables alone. This result was similar to the findings of Li et al. (2012) where texture attributes from L-band SAR data were less important than polarimetric attributes.

Nevertheless, the error rates obtained in the heuristic classification approach that aimed to gauge the separation of either closed or open canopy forests; or broadleaved, coniferous, and mixed forests were comparably much higher than using the multi-level hierarchical classification approach, which suggests that using a hierarchical clustering analysis to find natural groupings of forest types in the data yielded better classification accuracies compared to a heuristic method. Also, in the case of discriminating mangroves from other forest types, apparent classification errors (false positive) were produced as a result of the classifier determining elevation as the most parsimonious attribute for its discrimination (from closer inspection of the resulting decision trees).

The low accuracies achieved from discriminating different forest cover types may be attributed to the low sensitivity of data layers and feature attributes used in distinguishing the physical characteristics of the forest types being mapped. It may also be attributed to the incompatibility of discriminating forest attributes by the PALSAR sensor with the forest categories based on the FAO Global Forest Resources Assessment classification scheme. L-band radar backscatter was poorly sensitive to canopy attributes (e.g., closed canopy and open canopy) that characterise the FAO Global Forest Resources Assessment forest cover types. While other studies have indicated better performance of L-band radar backscatter at discriminating forest cover types characterised by forest structural attributes such as broadleaved and coniferous, the error rates obtained in this study suggested otherwise and also showed that error rates obtained for forest types described by forest canopy attributes were slightly lower than the error rates obtained for forests types described by structural attributes.
