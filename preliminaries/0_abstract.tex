%--------------------------------------------------------------------------
% Abstract
%--------------------------------------------------------------------------

Forests are important ecosystems that provide a broad range of goods and services, including social and economic benefits in the long term. To sustainably manage forest resources, remote sensing technologies are being employed as an accurate and cost-effective approach for mapping and monitoring forest conditions. In the Philippines, spaceborne remote sensing technology was similarly used in the past for forest resources assessment and management. The mapping efforts, however, were fraught with limitations in terms of temporal consistency and availability of optical data; inconsistent forest classification systems used; and replicability of approaches. This study evaluated the suitability of ALOS/PALSAR mosaic data for mapping and monitoring of forest cover types in the Philippines by assessing temporal consistency, forest classification hierarchy, and classification accuracy. Digital image processing approaches were used including object-based segmentation, extraction of texture features, hierarchical clustering, and decision tree classification. A combination of feature attributes including polarimetric, topographic, and texture information were assessed. Results showed that the ALOS/PALSAR mosaics were temporally consistent within a single acquisition year and across multiple acquisition years, making it a suitable data product for mapping and monitoring forests. Combining polarimetric, topographic, and texture attributes yielded better classification accuracies compared to using polarimetric SAR data alone, although the addition of ancillary feature attributes marginally decreased temporal consistency of the input data. Employing a decision tree classification approach identified the most relevant image layers and feature attributes for classifying different classes. Hierarchical multi-level classification provided a consistent approach for quantifying the degree of accuracy in discriminating different forest cover types. However, even with the best combination of feature attributes used in study, low classification accuracies were obtained in discriminating forest cover types. Although compared to heuristic methods, using a hierarchical clustering analysis to find natural groupings of forest types in the data yielded better classification accuracies. ALOS/PALSAR was not suitable for discriminating forest types based on the FAO Global Forest Resource Assessment classification scheme, and exploring other classification schemes is recommended.